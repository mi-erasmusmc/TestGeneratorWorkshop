% Options for packages loaded elsewhere
\PassOptionsToPackage{unicode}{hyperref}
\PassOptionsToPackage{hyphens}{url}
%
\documentclass[
]{article}
\usepackage{amsmath,amssymb}
\usepackage{iftex}
\ifPDFTeX
  \usepackage[T1]{fontenc}
  \usepackage[utf8]{inputenc}
  \usepackage{textcomp} % provide euro and other symbols
\else % if luatex or xetex
  \usepackage{unicode-math} % this also loads fontspec
  \defaultfontfeatures{Scale=MatchLowercase}
  \defaultfontfeatures[\rmfamily]{Ligatures=TeX,Scale=1}
\fi
\usepackage{lmodern}
\ifPDFTeX\else
  % xetex/luatex font selection
\fi
% Use upquote if available, for straight quotes in verbatim environments
\IfFileExists{upquote.sty}{\usepackage{upquote}}{}
\IfFileExists{microtype.sty}{% use microtype if available
  \usepackage[]{microtype}
  \UseMicrotypeSet[protrusion]{basicmath} % disable protrusion for tt fonts
}{}
\makeatletter
\@ifundefined{KOMAClassName}{% if non-KOMA class
  \IfFileExists{parskip.sty}{%
    \usepackage{parskip}
  }{% else
    \setlength{\parindent}{0pt}
    \setlength{\parskip}{6pt plus 2pt minus 1pt}}
}{% if KOMA class
  \KOMAoptions{parskip=half}}
\makeatother
\usepackage{xcolor}
\usepackage[margin=1in]{geometry}
\usepackage{color}
\usepackage{fancyvrb}
\newcommand{\VerbBar}{|}
\newcommand{\VERB}{\Verb[commandchars=\\\{\}]}
\DefineVerbatimEnvironment{Highlighting}{Verbatim}{commandchars=\\\{\}}
% Add ',fontsize=\small' for more characters per line
\usepackage{framed}
\definecolor{shadecolor}{RGB}{248,248,248}
\newenvironment{Shaded}{\begin{snugshade}}{\end{snugshade}}
\newcommand{\AlertTok}[1]{\textcolor[rgb]{0.94,0.16,0.16}{#1}}
\newcommand{\AnnotationTok}[1]{\textcolor[rgb]{0.56,0.35,0.01}{\textbf{\textit{#1}}}}
\newcommand{\AttributeTok}[1]{\textcolor[rgb]{0.13,0.29,0.53}{#1}}
\newcommand{\BaseNTok}[1]{\textcolor[rgb]{0.00,0.00,0.81}{#1}}
\newcommand{\BuiltInTok}[1]{#1}
\newcommand{\CharTok}[1]{\textcolor[rgb]{0.31,0.60,0.02}{#1}}
\newcommand{\CommentTok}[1]{\textcolor[rgb]{0.56,0.35,0.01}{\textit{#1}}}
\newcommand{\CommentVarTok}[1]{\textcolor[rgb]{0.56,0.35,0.01}{\textbf{\textit{#1}}}}
\newcommand{\ConstantTok}[1]{\textcolor[rgb]{0.56,0.35,0.01}{#1}}
\newcommand{\ControlFlowTok}[1]{\textcolor[rgb]{0.13,0.29,0.53}{\textbf{#1}}}
\newcommand{\DataTypeTok}[1]{\textcolor[rgb]{0.13,0.29,0.53}{#1}}
\newcommand{\DecValTok}[1]{\textcolor[rgb]{0.00,0.00,0.81}{#1}}
\newcommand{\DocumentationTok}[1]{\textcolor[rgb]{0.56,0.35,0.01}{\textbf{\textit{#1}}}}
\newcommand{\ErrorTok}[1]{\textcolor[rgb]{0.64,0.00,0.00}{\textbf{#1}}}
\newcommand{\ExtensionTok}[1]{#1}
\newcommand{\FloatTok}[1]{\textcolor[rgb]{0.00,0.00,0.81}{#1}}
\newcommand{\FunctionTok}[1]{\textcolor[rgb]{0.13,0.29,0.53}{\textbf{#1}}}
\newcommand{\ImportTok}[1]{#1}
\newcommand{\InformationTok}[1]{\textcolor[rgb]{0.56,0.35,0.01}{\textbf{\textit{#1}}}}
\newcommand{\KeywordTok}[1]{\textcolor[rgb]{0.13,0.29,0.53}{\textbf{#1}}}
\newcommand{\NormalTok}[1]{#1}
\newcommand{\OperatorTok}[1]{\textcolor[rgb]{0.81,0.36,0.00}{\textbf{#1}}}
\newcommand{\OtherTok}[1]{\textcolor[rgb]{0.56,0.35,0.01}{#1}}
\newcommand{\PreprocessorTok}[1]{\textcolor[rgb]{0.56,0.35,0.01}{\textit{#1}}}
\newcommand{\RegionMarkerTok}[1]{#1}
\newcommand{\SpecialCharTok}[1]{\textcolor[rgb]{0.81,0.36,0.00}{\textbf{#1}}}
\newcommand{\SpecialStringTok}[1]{\textcolor[rgb]{0.31,0.60,0.02}{#1}}
\newcommand{\StringTok}[1]{\textcolor[rgb]{0.31,0.60,0.02}{#1}}
\newcommand{\VariableTok}[1]{\textcolor[rgb]{0.00,0.00,0.00}{#1}}
\newcommand{\VerbatimStringTok}[1]{\textcolor[rgb]{0.31,0.60,0.02}{#1}}
\newcommand{\WarningTok}[1]{\textcolor[rgb]{0.56,0.35,0.01}{\textbf{\textit{#1}}}}
\usepackage{graphicx}
\makeatletter
\def\maxwidth{\ifdim\Gin@nat@width>\linewidth\linewidth\else\Gin@nat@width\fi}
\def\maxheight{\ifdim\Gin@nat@height>\textheight\textheight\else\Gin@nat@height\fi}
\makeatother
% Scale images if necessary, so that they will not overflow the page
% margins by default, and it is still possible to overwrite the defaults
% using explicit options in \includegraphics[width, height, ...]{}
\setkeys{Gin}{width=\maxwidth,height=\maxheight,keepaspectratio}
% Set default figure placement to htbp
\makeatletter
\def\fps@figure{htbp}
\makeatother
\setlength{\emergencystretch}{3em} % prevent overfull lines
\providecommand{\tightlist}{%
  \setlength{\itemsep}{0pt}\setlength{\parskip}{0pt}}
\setcounter{secnumdepth}{-\maxdimen} % remove section numbering
\ifLuaTeX
  \usepackage{selnolig}  % disable illegal ligatures
\fi
\usepackage{bookmark}
\IfFileExists{xurl.sty}{\usepackage{xurl}}{} % add URL line breaks if available
\urlstyle{same}
\hypersetup{
  pdftitle={Sample ICU Study Testing},
  hidelinks,
  pdfcreator={LaTeX via pandoc}}

\title{Sample ICU Study Testing}
\author{}
\date{\vspace{-2.5em}2025-01-30}

\begin{document}
\maketitle

\subsection{Setting Up the Test
Environment}\label{setting-up-the-test-environment}

Before running the tests, ensure that you have installed the necessary
packages. You can install them using the following command:

\begin{verbatim}
install.packages(c("dplyr", "CDMConnector", "omopgenerics", "testthat", "PatientProfiles", "here"))
\end{verbatim}

\subsection{\texorpdfstring{Running the Tests with
\texttt{testthat}}{Running the Tests with testthat}}\label{running-the-tests-with-testthat}

The following code executes three separate tests:

\begin{itemize}
\tightlist
\item
  \textbf{Checking cohort counts}
\item
  \textbf{Characterization}
\item
  \textbf{Running the study}
\end{itemize}

\subsection{Creating a Test Database Using
Excel}\label{creating-a-test-database-using-excel}

\begin{itemize}
\item
  This project includes a sample population dataset,
  \texttt{icu\_sample\_population}, located in the
  \texttt{tests/testthat} folder. It contains:

  \begin{itemize}
  \tightlist
  \item
    5 patients with ICU visits
  \item
    7 patients with COVID-19
  \item
    5 patients with mechanical ventilation
  \end{itemize}

  Explore the dataset and modify the tables to create your own test
  cases.
\item
  The \texttt{TestGenerator} package provides a blank CDM, allowing you
  to input your own patient data from scratch.
\item
  You can also create a test database for one of your ongoing studies.
\item
  If needed, you can collaborate with a data scientist, or we can assist
  in processing the patients and running the tests for you.
\end{itemize}

\begin{Shaded}
\begin{Highlighting}[]
\NormalTok{cdmVersion }\OtherTok{\textless{}{-}} \StringTok{"5.3"}

\NormalTok{TestGenerator}\SpecialCharTok{::}\FunctionTok{generateTestTables}\NormalTok{(}
  \AttributeTok{tableNames =} \FunctionTok{c}\NormalTok{(}\StringTok{"person"}\NormalTok{, }\StringTok{"drug\_exposure"}\NormalTok{, }\StringTok{"condition\_occurrence"}\NormalTok{),}
  \AttributeTok{cdmVersion =}\NormalTok{ cdmVersion,}
  \AttributeTok{outputFolder =}\NormalTok{ here}\SpecialCharTok{::}\FunctionTok{here}\NormalTok{(}\StringTok{"tests"}\NormalTok{, }\StringTok{"testthat"}\NormalTok{),}
  \AttributeTok{filename =} \FunctionTok{paste0}\NormalTok{(}\StringTok{"test\_cdm\_"}\NormalTok{, cdmVersion)}
\NormalTok{)}
\end{Highlighting}
\end{Shaded}

\subsection{Reading the Test Data}\label{reading-the-test-data}

After generating the test patients, a data scientist can use the Excel
file to create the test CDM.

\begin{Shaded}
\begin{Highlighting}[]
\NormalTok{filePath }\OtherTok{\textless{}{-}}\NormalTok{ testthat}\SpecialCharTok{::}\FunctionTok{test\_path}\NormalTok{(}\StringTok{"icu\_sample\_population.xlsx"}\NormalTok{)}
\NormalTok{TestGenerator}\SpecialCharTok{::}\FunctionTok{readPatients.xl}\NormalTok{(}
  \AttributeTok{filePath =}\NormalTok{ filePath,}
  \AttributeTok{testName =} \StringTok{"icu\_sample\_population"}\NormalTok{,}
  \AttributeTok{cdmVersion =} \StringTok{"5.3"}
\NormalTok{)}
\end{Highlighting}
\end{Shaded}

\begin{verbatim}
## v Unit Test Definition Created Successfully: 'icu_sample_population'
\end{verbatim}

\subsection{Creating the CDM Database in
Memory}\label{creating-the-cdm-database-in-memory}

Now, we can select a dataset to create our CDM reference.

\begin{Shaded}
\begin{Highlighting}[]
\NormalTok{cdm }\OtherTok{\textless{}{-}}\NormalTok{ TestGenerator}\SpecialCharTok{::}\FunctionTok{patientsCDM}\NormalTok{(}\AttributeTok{testName =} \StringTok{"icu\_sample\_population"}\NormalTok{)}
\end{Highlighting}
\end{Shaded}

\begin{verbatim}
## Note: method with signature 'DBIConnection#Id' chosen for function 'dbExistsTable',
##  target signature 'duckdb_connection#Id'.
##  "duckdb_connection#ANY" would also be valid
\end{verbatim}

\begin{verbatim}
## ! cdm name not specified and could not be inferred from the cdm source table
\end{verbatim}

\begin{verbatim}
## v Patients pushed to blank CDM successfully
\end{verbatim}

\subsection{Exploring the Data as a Standard CDM
Reference}\label{exploring-the-data-as-a-standard-cdm-reference}

The CDM reference contains all tables following standard conventions.

\begin{Shaded}
\begin{Highlighting}[]
\NormalTok{cdm}
\end{Highlighting}
\end{Shaded}

\begin{verbatim}
## 
\end{verbatim}

\begin{verbatim}
## -- # OMOP CDM reference (duckdb) of An OMOP CDM database -----------------------
\end{verbatim}

\begin{verbatim}
## * omop tables: person, observation_period, visit_occurrence, visit_detail,
## condition_occurrence, drug_exposure, procedure_occurrence, device_exposure,
## measurement, observation, death, note, note_nlp, specimen, fact_relationship,
## location, care_site, provider, payer_plan_period, cost, drug_era, dose_era,
## condition_era, metadata, cdm_source, concept, vocabulary, domain,
## concept_class, concept_relationship, relationship, concept_synonym,
## concept_ancestor, source_to_concept_map, drug_strength, cohort_definition,
## attribute_definition
\end{verbatim}

\begin{verbatim}
## * cohort tables: -
\end{verbatim}

\begin{verbatim}
## * achilles tables: -
\end{verbatim}

\begin{verbatim}
## * other tables: -
\end{verbatim}

\begin{Shaded}
\begin{Highlighting}[]
\NormalTok{cdm}\SpecialCharTok{$}\NormalTok{person}
\end{Highlighting}
\end{Shaded}

\begin{verbatim}
## # Source:   table<main.person> [?? x 18]
## # Database: DuckDB v1.1.3 [cbarboza@Windows 10 x64:R 4.4.1/C:\Users\cbarboza\AppData\Local\Temp\RtmpMnFyHA\file1f181449557f.duckdb]
##   person_id gender_concept_id year_of_birth month_of_birth day_of_birth
##       <int>             <int>         <int>          <int>        <int>
## 1         1              8532          1980             NA           NA
## 2         2              8507          1990             NA           NA
## 3         3              8532          2000             NA           NA
## 4         4              8507          1980             NA           NA
## 5         5              8532          1990             NA           NA
## 6         6              8507          2000             NA           NA
## 7         7              8532          1980             NA           NA
## 8         8              8507          1990             NA           NA
## # i 13 more variables: birth_datetime <dttm>, race_concept_id <int>,
## #   ethnicity_concept_id <int>, location_id <int>, provider_id <int>,
## #   care_site_id <int>, person_source_value <chr>, gender_source_value <chr>,
## #   gender_source_concept_id <int>, race_source_value <chr>,
## #   race_source_concept_id <int>, ethnicity_source_value <chr>,
## #   ethnicity_source_concept_id <int>
\end{verbatim}

\subsection{\texorpdfstring{Creating Cohorts and Running Tests in the
\texttt{testthat}
Environment}{Creating Cohorts and Running Tests in the testthat Environment}}\label{creating-cohorts-and-running-tests-in-the-testthat-environment}

The following steps are performed:

\begin{itemize}
\tightlist
\item
  Checking cohort counts
\item
  Testing the characterization function
\item
  Running the study
\end{itemize}

\begin{Shaded}
\begin{Highlighting}[]
\FunctionTok{library}\NormalTok{(dplyr)}
\FunctionTok{library}\NormalTok{(CDMConnector)}
\FunctionTok{library}\NormalTok{(omopgenerics)}
\FunctionTok{library}\NormalTok{(testthat)}
\FunctionTok{library}\NormalTok{(PatientProfiles)}
\FunctionTok{library}\NormalTok{(here)}
\FunctionTok{source}\NormalTok{(here}\SpecialCharTok{::}\FunctionTok{here}\NormalTok{(}\StringTok{"R/icu.R"}\NormalTok{))}

\FunctionTok{test\_that}\NormalTok{(}\StringTok{"Check cohort counts for COVID, ICU visits, and ventilation"}\NormalTok{, \{}
\NormalTok{  cdm }\OtherTok{\textless{}{-}}\NormalTok{ TestGenerator}\SpecialCharTok{::}\FunctionTok{patientsCDM}\NormalTok{(}\AttributeTok{testName =} \StringTok{"icu\_sample\_population"}\NormalTok{)}
  
  \CommentTok{\# Generate cohorts}
\NormalTok{  cohorts }\OtherTok{\textless{}{-}}\NormalTok{ here}\SpecialCharTok{::}\FunctionTok{here}\NormalTok{(}\StringTok{"cohorts"}\NormalTok{, }\StringTok{"population"}\NormalTok{)}
\NormalTok{  icu\_cohort\_sets }\OtherTok{\textless{}{-}}\NormalTok{ CDMConnector}\SpecialCharTok{::}\FunctionTok{readCohortSet}\NormalTok{(cohorts)}
\NormalTok{  cdm }\OtherTok{\textless{}{-}}\NormalTok{ CDMConnector}\SpecialCharTok{::}\FunctionTok{generate\_cohort\_set}\NormalTok{(cdm, icu\_cohort\_sets, }\AttributeTok{name =} \StringTok{"icu\_population\_cohorts"}\NormalTok{)}
\NormalTok{  cdm[[}\StringTok{"icu\_population\_cohorts"}\NormalTok{]] }\SpecialCharTok{\%\textgreater{}\%}\NormalTok{ omopgenerics}\SpecialCharTok{::}\FunctionTok{settings}\NormalTok{()}
  
  \CommentTok{\# Check counts}
\NormalTok{  cohort\_counts }\OtherTok{\textless{}{-}}\NormalTok{ omopgenerics}\SpecialCharTok{::}\FunctionTok{cohortCount}\NormalTok{(cdm}\SpecialCharTok{$}\NormalTok{icu\_population\_cohorts)}
  
  \FunctionTok{expect\_equal}\NormalTok{(cohort\_counts[}\DecValTok{1}\NormalTok{, ] }\SpecialCharTok{\%\textgreater{}\%} \FunctionTok{pull}\NormalTok{(number\_records), }\DecValTok{5}\NormalTok{)}
  \FunctionTok{expect\_equal}\NormalTok{(cohort\_counts[}\DecValTok{2}\NormalTok{, ] }\SpecialCharTok{\%\textgreater{}\%} \FunctionTok{pull}\NormalTok{(number\_records), }\DecValTok{7}\NormalTok{)}
  \FunctionTok{expect\_equal}\NormalTok{(cohort\_counts[}\DecValTok{3}\NormalTok{, ] }\SpecialCharTok{\%\textgreater{}\%} \FunctionTok{pull}\NormalTok{(number\_records), }\DecValTok{5}\NormalTok{)}
\NormalTok{\})}

\FunctionTok{test\_that}\NormalTok{(}\StringTok{"Characterization"}\NormalTok{, \{}
\NormalTok{  cdm }\OtherTok{\textless{}{-}}\NormalTok{ TestGenerator}\SpecialCharTok{::}\FunctionTok{patientsCDM}\NormalTok{(}\AttributeTok{testName =} \StringTok{"icu\_sample\_population"}\NormalTok{)}
\NormalTok{  cohorts }\OtherTok{\textless{}{-}}\NormalTok{ here}\SpecialCharTok{::}\FunctionTok{here}\NormalTok{(}\StringTok{"cohorts"}\NormalTok{, }\StringTok{"population"}\NormalTok{)}
  
\NormalTok{  icu\_cohort }\OtherTok{\textless{}{-}}\NormalTok{ CDMConnector}\SpecialCharTok{::}\FunctionTok{readCohortSet}\NormalTok{(cohorts) }\SpecialCharTok{\%\textgreater{}\%}
\NormalTok{    dplyr}\SpecialCharTok{::}\FunctionTok{filter}\NormalTok{(cohort\_name }\SpecialCharTok{==} \StringTok{"p2\_c1\_014\_icu\_visit\_dv\_v2"}\NormalTok{)}
\NormalTok{  cdm }\OtherTok{\textless{}{-}}\NormalTok{ CDMConnector}\SpecialCharTok{::}\FunctionTok{generate\_cohort\_set}\NormalTok{(cdm, icu\_cohort, }\AttributeTok{name =} \StringTok{"icu"}\NormalTok{)}
  
\NormalTok{  covid\_cohort }\OtherTok{\textless{}{-}}\NormalTok{ CDMConnector}\SpecialCharTok{::}\FunctionTok{readCohortSet}\NormalTok{(cohorts) }\SpecialCharTok{\%\textgreater{}\%}
\NormalTok{    dplyr}\SpecialCharTok{::}\FunctionTok{filter}\NormalTok{(cohort\_name }\SpecialCharTok{==} \StringTok{"p2\_c1\_014\_covid19\_v2"}\NormalTok{)}
\NormalTok{  cdm }\OtherTok{\textless{}{-}}\NormalTok{ CDMConnector}\SpecialCharTok{::}\FunctionTok{generate\_cohort\_set}\NormalTok{(cdm, covid\_cohort, }\AttributeTok{name =} \StringTok{"covid"}\NormalTok{)}
  
\NormalTok{  ventilation\_cohort }\OtherTok{\textless{}{-}}\NormalTok{ CDMConnector}\SpecialCharTok{::}\FunctionTok{readCohortSet}\NormalTok{(cohorts) }\SpecialCharTok{\%\textgreater{}\%}
\NormalTok{    dplyr}\SpecialCharTok{::}\FunctionTok{filter}\NormalTok{(cohort\_name }\SpecialCharTok{==} \StringTok{"p2\_c1\_014\_mechanical\_ventilation\_dv\_v2"}\NormalTok{)}
\NormalTok{  cdm }\OtherTok{\textless{}{-}}\NormalTok{ CDMConnector}\SpecialCharTok{::}\FunctionTok{generate\_cohort\_set}\NormalTok{(cdm, ventilation\_cohort, }\AttributeTok{name =} \StringTok{"ventilation"}\NormalTok{)}
  
\NormalTok{  summaryCharacterization }\OtherTok{\textless{}{-}} \FunctionTok{characterizeICU}\NormalTok{(}
    \AttributeTok{cdm =}\NormalTok{ cdm,}
    \AttributeTok{icuCohort =} \StringTok{"icu"}\NormalTok{,}
    \AttributeTok{covidCohort =} \StringTok{"covid"}\NormalTok{,}
    \AttributeTok{ventilationCohort =} \StringTok{"ventilation"}
\NormalTok{  )}
  
  \FunctionTok{expect\_s3\_class}\NormalTok{(summaryCharacterization, }\FunctionTok{c}\NormalTok{(}\StringTok{"summarised\_result"}\NormalTok{, }\StringTok{"omop\_result"}\NormalTok{))}
  \FunctionTok{expect\_equal}\NormalTok{(}
\NormalTok{    summaryCharacterization }\SpecialCharTok{\%\textgreater{}\%} 
      \FunctionTok{filter}\NormalTok{(variable\_name }\SpecialCharTok{==} \StringTok{"covid\_m14\_to\_0"}\NormalTok{, estimate\_name }\SpecialCharTok{==} \StringTok{"count"}\NormalTok{) }\SpecialCharTok{\%\textgreater{}\%} 
      \FunctionTok{pull}\NormalTok{(estimate\_value), }\StringTok{"4"}
\NormalTok{  )}
  \FunctionTok{expect\_equal}\NormalTok{(}
\NormalTok{    summaryCharacterization }\SpecialCharTok{\%\textgreater{}\%} 
      \FunctionTok{filter}\NormalTok{(variable\_name }\SpecialCharTok{==} \StringTok{"ventilation\_m7\_to\_0"}\NormalTok{, estimate\_name }\SpecialCharTok{==} \StringTok{"count"}\NormalTok{) }\SpecialCharTok{\%\textgreater{}\%} 
      \FunctionTok{pull}\NormalTok{(estimate\_value), }\StringTok{"3"}
\NormalTok{  )}
\NormalTok{\})}

\FunctionTok{test\_that}\NormalTok{(}\StringTok{"Run Study"}\NormalTok{, \{}
\NormalTok{  cdm }\OtherTok{\textless{}{-}}\NormalTok{ TestGenerator}\SpecialCharTok{::}\FunctionTok{patientsCDM}\NormalTok{(}\AttributeTok{testName =} \StringTok{"icu\_sample\_population"}\NormalTok{)}
\NormalTok{  outputDir }\OtherTok{\textless{}{-}} \FunctionTok{paste0}\NormalTok{(}\FunctionTok{tempdir}\NormalTok{(), }\StringTok{"/results"}\NormalTok{)}
  \FunctionTok{dir.create}\NormalTok{(outputDir)}
  
  \FunctionTok{runStudy}\NormalTok{(cdm, }\AttributeTok{outputDir =}\NormalTok{ outputDir)}
  
\NormalTok{  files\_created }\OtherTok{\textless{}{-}} \FunctionTok{list.files}\NormalTok{(outputDir)}
  \FunctionTok{expect\_true}\NormalTok{(}\StringTok{"icu\_summary.csv"} \SpecialCharTok{\%in\%}\NormalTok{ files\_created)}
  
  \FunctionTok{unlink}\NormalTok{(outputDir, }\AttributeTok{recursive =} \ConstantTok{TRUE}\NormalTok{)}
\NormalTok{\})}
\end{Highlighting}
\end{Shaded}


\end{document}
